\documentclass[12pt]{article}

%%%% packages and definitions (optional)
\usepackage{graphicx} % allows inclusion of graphics
\usepackage{graphics}
\usepackage{placeins}
\usepackage{booktabs} % nice rules (thick lines) for tables
\usepackage{microtype} % improves typography for PDF
\usepackage{xspace}
\usepackage[hidelinks]{hyperref}
\usepackage{xspace}
\usepackage{hhline}
\usepackage{amsmath}

\usepackage{tabularx}
\newcolumntype{b}{>{\hsize=1.0\hsize}X}
\newcolumntype{s}{>{\hsize=.5\hsize}X}
\newcolumntype{m}{>{\hsize=.75\hsize}X}

\newcommand{\SN}{S$_N$}
\renewcommand{\vec}[1]{\bm{#1}} %vector is bold italic
\newcommand{\vd}{\bm{\cdot}} % slightly bold vector dot
\newcommand{\grad}{\vec{\nabla}} % gradient
\newcommand{\ud}{\mathop{}\!\mathrm{d}} % upright derivative symbol
\newcommand{\Cyclus}{\textsc{Cyclus}\xspace}%
\graphicspath{ {images/} }
\usepackage[affil-it]{authblk}
\usepackage[numbers]{natbib}
\usepackage{notoccite}
\usepackage{tikz}
\usetikzlibrary{positioning, arrows, decorations, shapes }
\usepackage{cleveref}

\usepackage{datatool}
\usepackage[acronym,toc]{glossaries}
\include{acros}
	
\makeglossaries

\title{Probabilistic Risk Assessment of Mined Nuclear Spent Fuel Repositories }
\author{Jin Whan Bae}
\affil{Dept. of Nuclear, Plasma, and Radiological Engineering, University of Illinois at Urbana-Champaign
		  Urbana, IL}
\date{}                     %% if you don't need date to appear
\setcounter{Maxaffil}{0}
\renewcommand\Affilfont{\itshape\small}
%%%%%%%%%%%%%%actual words%%%%%%%%%%%%%%%%%%%%%%%%%%%%%%%%%%%%%%%%%%%%%%%%%%%%5


\begin{document}
\maketitle

\section{Abstract}
\gls{UNF} repositories, given the high decay heat and radioactivity
of \glspl{UNF}, requires careful engineering. The current plan is
to design a repository that would contain the material for one million years.
Considering various events (failures) can occur in that time range,
the \gls{UNF} repository proposes an interesting subject for
\gls{PRA}. In this report, the model repository design is after the 
Yucca Mountain Repository, which is a mined repository in volcanic tuff.
Ultimate failure status can be defined in various ways, depending on the
extent of leakage during the one million years (leakage from canister
\textasciitilde exposure to nearest population). Given the expansive
time range, very little real data is available, which makes
most of the probability values a Bayesian, `degree-of-confidence'
value. Some data can be derived from scientific data, such as 
corrosion and precipitation in the area, but would have to be extrapolated
with assumptions for the given time range. The failure scenarios
would start from the complete placement of the canisters, assuming
the \gls{UNF} canisters arrived and was placed in the repository
without fail.


%\bibliographystyle{unsrtnat}
%\bibliography{bibliography}


\end{document}
\grid
