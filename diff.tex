
\section{Fundamental Code Differences in \Cyclus}

\Cyclus has fundamental code differences than some fuel cycle analysis codes
used in the benchmark \cite{feng_standardized_2016}.

\Cyclus has a default timestep of a month. In order to take
this into account, we calculate the annual value
(i.e. average value for inventory values and sum of 12 months
for throughput values) for each result. The timestep in \Cyclus
can be changed into a year without changing the source
code, but \Cyclus' timestep execution (figure \ref{fig:time}) causes a delay
in the material flow. Thus, having the timestep be 12 months
allows the lessening of the impact of the delay due to the
\Cyclus timestep execution.

Similarly, \Cyclus has discrete
execution steps per timestep that might cause delays or contort
the results from other simulators. For example, decommissioning of
facilities occur at the end of a timestep, while building of facilities
occur at the beginning of a timestep.

The \Cycamore recipe reactor depletes half of its core when decommissioned,
whereas the codes in the benchmark \cite{feng_standardized_2016} deplete all its fuel when decommissioned. This causes a major
discrepancy for \gls{TRU} inventory. For this study, we changed
the \Cycamore source code to deplete all its assemblies to the depleted recipe.
Also, the \Cycamore recipe reactor treats each batch (and assembly) as a discrete
material, while some codes have continuous fuel discharge. This produces
differences in the results because the batches in the benchmark \cite{feng_standardized_2016} are in fractions.
In this study, the \gls{LWR} batch size and cycle time are increased, while
decreasing the batch number to keep the core size constant. We simply round
up the \gls{SFR} batch number, while the batch size and cycle time are kept constant.
This increases the core size by $1.08 \%$, which is negligible, but will be
discussed in the results section.
We list the differences in table \ref{tab:diff}.

\begin{table}[h]
    \centering
    \caption{Difference in Batch number and core size}
\begin{tabularx}{0.47\textwidth}{bss}
        \hline
        \textbf{Category} & \textbf{Benchmark \cite{feng_standardized_2016}} & \textbf{This study} \\
        \hline
        LWR Batches & 4.5 & 3 \\
        LWR Batch size [tHM] & 19.91 & 29.86 \\
        LWR Core size [tHM] & 89.59 & 89.59 \\
        LWR Cycle time & 1 year & 1.5 years \\
        SFR Batches & 3.96 & 4 \\
        SFR Batch size [tHM] & 3.95 & 3.95 \\
        SFR Core size [tHM] & 15.63 & 15.8 \\
        \hline
        \end{tabularx}
        \label{tab:diff}
\end {table}

Note that all the differences could have been mediated by changing the
archetype source codes. However, the only change made was the reactor
depletion behavior at decommissioning due to its large impact. Note that the
goal of this
study is to show current \Cyclus agreement with other codes and identify
differences, not to alter \Cyclus to match the other codes.